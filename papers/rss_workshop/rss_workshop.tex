\documentclass[conference]{IEEEtran}
\usepackage{times}
\usepackage{algorithm}
\usepackage{algpseudocode}
\usepackage{graphicx}
\usepackage{subfloat}
\usepackage{amsmath,empheq}

% numbers option provides compact numerical references in the text. 
\usepackage[numbers]{natbib}
\usepackage{multicol}
\usepackage[bookmarks=true]{hyperref}

\pdfinfo{
   /Author (David Abel  & Gabriel Barth-Maron, James MacGlashan, Stefanie Tellex)
   /Title  (Affordace-Aware Planning)
   /CreationDate (D:20101201120000)
   /Subject (Planning, Affordances, Sequential Decision Making)
   /Keywords (Planning, Affordance, MDP, Learning)
}

\begin{document}

% paper title
\title{Affordance-Aware Planning}

% You will get a Paper-ID when submitting a pdf file to the conference system
\author{Paper-ID [add your ID here]}

%\author{\authorblockN{Michael Shell}
%\authorblockA{School of Electrical and\\Computer Engineering\\
%Georgia Institute of Technology\\
%Atlanta, Georgia 30332--0250\\
%Email: mshell@ece.gatech.edu}
%\and
%\authorblockN{Homer Simpson}
%\authorblockA{Twentieth Century Fox\\
%Springfield, USA\\
%Email: homer@thesimpsons.com}
%\and
%\authorblockN{James Kirk\\ and Montgomery Scott}
%\authorblockA{Starfleet Academy\\
%San Francisco, California 96678-2391\\
%Telephone: (800) 555--1212\\
%Fax: (888) 555--1212}}


% avoiding spaces at the end of the author lines is not a problem with
% conference papers because we don't use \thanks or \IEEEmembership


% for over three affiliations, or if they all won't fit within the width
% of the page, use this alternative format:
% 
%\author{\authorblockN{Michael Shell\authorrefmark{1},
%Homer Simpson\authorrefmark{2},
%James Kirk\authorrefmark{3}, 
%Montgomery Scott\authorrefmark{3} and
%Eldon Tyrell\authorrefmark{4}}
%\authorblockA{\authorrefmark{1}School of Electrical and Computer Engineering\\
%Georgia Institute of Technology,
%Atlanta, Georgia 30332--0250\\ Email: mshell@ece.gatech.edu}
%\authorblockA{\authorrefmark{2}Twentieth Century Fox, Springfield, USA\\
%Email: homer@thesimpsons.com}
%\authorblockA{\authorrefmark{3}Starfleet Academy, San Francisco, California 96678-2391\\
%Telephone: (800) 555--1212, Fax: (888) 555--1212}
%\authorblockA{\authorrefmark{4}Tyrell Inc., 123 Replicant Street, Los Angeles, California 90210--4321}}


\maketitle

\begin{abstract}
Planning algorithms for non-deterministic domains are often
intractable in large state spaces due to the well-known ``curse of
dimensionality.'' Existing approaches to address this problem fail to
prevent the planner from considering many actions which would be
obviously irrelevant to a human solving the same problem. We
introduce a novel, state- and reward- general approach to pruning
actions while solving an MDP by encoding knowledge about the
domain in terms of {\em affordances}~\citep{gibson77}. This pruning 
significantly reduces the number of state-action pairs the agent needs 
to evaluate in order to act optimally. We demonstrate our approach 
in the Minecraft domain, showing significant increase in speed and 
reduction in state-space exploration compared to the standard 
versions of these algorithms. Further, we provide a learning framework
that enables an agent to learn affordances through experience, removing 
the dependence on the expert. We provide preliminary results indicating that the learning
process effectively produces affordances that help solve an MDP faster.
\end{abstract}

\IEEEpeerreviewmaketitle

% ====== Section: Introduction ======
\section{Introduction}
\label{sec:introduction}
As robots move out of the lab and into the real world, planning
algorithms need to scale to domains of increased noise, size, and
complexity.  A classic formalization of this problem is a stochastic
sequential decision making problem in which the agent must find a
policy (a mapping from states to actions) for some subset of the state
space that enables the agent to achieve a goal from some initial
state, while minimizing any costs along the way.
Increases in planning problem size and complexity directly correspond
to an explosion in the state-action space. Current approaches to solving 
sequential decision making problems in the face of uncertainty cannot tackle these problems 
as the state-action space becomes too large~\citep{grounds05}.

To address this state-space explosion, prior work has explored adding
knowledge to the planner to solve problems in these
massive domains, such as options~\citep{sutton99} and
macroactions~\citep{Botea:2005kx,Newton:2005vn}. However, these
approaches add knowledge in the form of additional high-level actions
to the agent, which {\em increases} the size of the state-action space
(while also allowing the agent to search more deeply within the
space).  The resulting augmented space is even larger, which can have
the paradoxical effect of increasing the search time for a good
policy. Further, other approaches fall short of learning useful, transferable knowledge,
either due to complexity or lack of generalizability (cite? where is this stated? George?).

Instead, we propose a formalization of {\em affordances} \citep{gibson77} that enables an agent to focus on
problem-specific aspects of the environment. Our approach avoids exploration of irrelevant parts of the 
state-action space, which leads to dramatic speedups in planning.

We formalize the notion of an affordance as a piece of planning
knowledge provided to an agent operating in a Markov Decision
Process (MDP). Affordances are not specific to a particular reward 
function or state space, and thus, provide the agent with transferable 
knowledge that is effective in a wide variety of problems. Because affordances
define the {\em kind} of goals for which actions are useful,
affordances also enable high-level reasoning that can
be combined with approaches like high-level subgoal planning for even
greater performance gains. Further, we propose a learning process that
enables agents to autonomously learn affordances through experience, lessening
the dependence on expert knowledge.


% ====== Section: Background ======
\section{Background}
\label{sec:background}
We use Minecraft as our planning and evaluation domain. Minecraft is a
3-D blocks world game in which the user can place and destroy blocks
of different types.  Minecraft's physics and action space is expressive
enough to allow very complex worlds to be created by users, such as a
functional scientific graphing calculator\footnote{https://www.youtube.com/watch?v=wgJfVRhotlQ};
simple scenes from a Minecraft world appear in Figure~\ref{fig:minecraft}.  

Minecraft serves as an effective parallel for the actual world, both
in terms of approximating the complexity and scope of planning
problems, as well as modeling the uncertainty and noise presented to a
real world agent.  For instance, robotic agents are prone to
uncertainty all throughout their system, including noise in their
sensors (cameras, LIDAR, microphones, etc.), odometry, control, and
actuation.  In order to accurately capture some of the inherent
difficulties of planning under uncertainty, the Minecraft agent's
actions were modified to have stochastic outcomes. These stochastic
outcomes may require important changes in the optimal policy in
contrast to deterministic actions, such as keeping the agent's
distance from a pit of lava. We chose to give the Minecraft agent perfect
sensor data about the Minecraft world, as that is presently beyond the focus of this
work.

\subsection{OO-MDPs}
We define affordances in terms of propositional functions on states. Our definition builds on the Object-Oriented Markov Decision Process
(OO-MDP)~\citep{diuk08}.  OO-MDPs are an extension of
the classic Markov Decision Process (MDP).  A classic MDP is a
five-tuple: $\langle \mathcal{S}, \mathcal{A}, \mathcal{T},
\mathcal{R}, \gamma \rangle$, where $\mathcal{S}$ is a state-space;
$\mathcal{A}$ is the agent's set of actions; $\mathcal{T}$ denotes
$\mathcal{T}(s' \mid s,a)$, the transition probability of an agent
applying action $a \in \mathcal{A}$ in state $s \in \mathcal{S}$ and
arriving in $s' \in \mathcal{S}$; $\mathcal{R}(s,a,s')$ denotes the
reward received by the agent for applying action $a$ in state $s$ and
and transitioning to state $s'$; and $\gamma \in [0, 1)$ is a discount
  factor that defines how much the agent prefers immediate rewards
  over distant rewards (the agent more greatly prefers to maximize
  more immediate rewards as $\gamma$ decreases).

A classic way to provide a factored representation of an MDP state is to represent
each MDP state as a single feature vector. By contrast, an OO-MDP represents the state space as a collection of objects,
$O = \{o_1, \ldots, o_o \}$.  Each object $o_i$ belongs to a
class $c_j \in  \{c_1, \ldots, c_c\}$. Every class has a set of attributes
$Att(c) = \{c.a_1, \ldots, c.a_a \}$, each of which has a domain $Dom(c.a)$ of possible values.
Upon instantiation of an object class, its attributes are given a state $o.state$
(an assignment of values to its attributes).  The underlying MDP state is the set
of all the object states: $s \in {\cal S} = \cup_{i = 1}^o \{o_i.state\}$.

There are two advantages to using an object-oriented factored state
representation instead of a single feature vector. First, different
states in the same state space may contain different numbers of
objects of varying classes, which is useful in domains like Minecraft
in which the agent can dynamically add and remove blocks to the
world. Second, MDP states can be defined invariantly to the specific
object references.  For instance, consider a Minecraft world with two
block objects, $b_1$ and $b_2$.  If the agent picked up and swapped
the position of $b_1$ and $b_2$, the MDP state before the swap and
after the swap would be the same, because the MDP state definition is
invariant to which object holds which object state. This
object reference invariance results in a smaller state space compared
to representations like feature vectors in which changes to value
assignments always result in a different state.

While the OO-MDP state definition is a good fit for the Minecraft
domain, our motivation for using an OO-MDP lies in the ability to
formulate predicates over classes of objects. That is, the OO-MDP
definition also includes a set of predicates ${\cal P}$ that operate
on the state of objects to provide additional high-level information
about the MDP state. For example, in \texttt{BRIDGEWORLD}, a ${\tt
  nearTrench}({\tt STATE})$ predicate evaluates to true when the singular
instance of class $\texttt{AGENT}$ is directly adjacent to an empty location
at floor level (i.e. the cell beneath the agent in some direction does not
contain a block). In the original OO-MDP work, these predicates were used
to model and learn an MDP's transition dynamics.


% ====== Section: Expert-Affordances ======
\section{Expert-Affordances}
\label{sec:expert-affordances}

We define an affordance $\Delta$ 
as the mapping $\langle p,g\rangle \longmapsto \mathcal{A}'$,
where:
\begin{itemize}
\item[] $\mathcal{A}'$ a subset of the action space, $\mathcal{A}$, representing the relevant {\it action-possibilities} of the environment.
\item[] $p$ is a predicate on states, $s \longrightarrow \{$0$, 1\}$
  representing the {\em precondition} for the affordance.
\item[] $g$ is an ungrounded predicate on states, $g$, representing a {\it lifted goal description}.
\end{itemize}
The precondition and goal description predicates refer to predicates that are defined in the OO-MDP definition. 
Using OO-MDP predicates for affordance preconditions and goal descriptions 
allows for state space independence. Thus, a planner equipped with
affordances can be used in any number of different tasks. For instance, the affordances defined for Minecraft navigation problems can be used in any task regardless of the spatial size of the world, number of blocks in the world, and specific goal location that needs to be reached.

We call any planner that uses affordances and {\it affordance-aware} planner.

Given a set of $n$ domain affordances $Z = \{\Delta_1, ..., \Delta_n\}$ and a current agent goal condition defined with an OO-MDP predicate $G$, the action set that a planning algorithm considers may be pruned on a state by state basis as shown in Algorithm~\ref{alg:prune_actions}.
\begin{algorithm}
  \caption{getActionsForState($state$, $Z$, $G$)}
  \begin{algorithmic}[1]
    \State $\mathcal{A}^* \leftarrow \{\}$
    \For {$\Delta \in Z$}
    \If {$\Delta.p(state)$ and $\Delta.g = G$}
    \State $\mathcal{A}^* \leftarrow \mathcal{A}^* \cup \Delta.\mathcal{A}'$
    \EndIf
    \EndFor \\
    \Return $\mathcal{A}^*$
  \end{algorithmic}
  \label{alg:prune_actions}
\end{algorithm}

Specifically, the algorithm starts
by initializing an empty set of actions $\mathcal{A}^*$ (line 1). The algorithm then iterates
through each of the domain affordances (lines 2-6). If the affordance
precondition ($\Delta.p$) is satisfied by some set of objects in the current state
and the affordance goal condition ($\Delta.g$) is defined with the same predicate
as the current goal (line 3), then the actions associated with the affordance ($\Delta.\mathcal{A}'$) are added to the action set $\mathcal{A}^*$ (line 4). Finally, $\mathcal{A}^*$ is returned (line 7). 

\subsection{Experiments}

We conducted a series of experiments in the Minecraft domain that
compared the performance of several OO-MDP solvers without affordances,
to their affordance-aware counterparts. We selected the expert
affordances from our background knowledge of the domain.  

\begin{figure}
\begin{empheq}{align*}
\vspace{6 pt}
\Delta_1 &= \langle nearTrench, reachGoal \rangle \longmapsto \{place, jump\} \\
\Delta_2 &= \langle onPlane, reachGoal \rangle \longmapsto \{move\} \\
\Delta_3 &= \langle nearWall,reachGoal \rangle \longmapsto \{destroy \} \\
\Delta_4 &= \langle nearFurnace, makeGold \rangle \longmapsto \{place\} \\
\Delta_5 &= \langle nearOre, makeGold \rangle \longmapsto \{destroy\}
\vspace{6 pt}
\end{empheq}
\caption{The five affordance types used in expert experiments.}
\label{fig:afford_kb_exp}
\end{figure}

We gave the agent a single knowledge base of 5 types of affordances, which are listed in Figure \ref{fig:afford_kb_exp}.
Our experiments consisted of a variety of common tasks in Minecraft, ranging from basic path planning, to smelting gold,
to opening doors and tunneling through walls.  We also tested each
planner on worlds of varying size and difficulty to demonstrate the
scalability and flexibility of the affordance formalism. The
evaluation metric for each trial was the number of state backups that
were executed by each planning algorithm. Value
Iteration was terminated when the maximum change in the value function
was less than 0.01. RTDP terminated when the maximum change in the
value function was less than 0.01 for five consecutive policy
rollouts. In subgoal planning, the high-level subgoal plan was solved
using breadth-first search; which only took a small fraction of the
time compared to the total low-level planning and therefore is not
reported. RTDP was used as the low-level planner for subgoal planning.

We set the reward function to $-1$ for all transitions, except
transitions to states in which the agent was on lava, which returned 
$-200$. The goal was set to be terminal. The discount
factor was set to $\lambda = 0.99$. For all experiments, the agent was given stochastic actions. Specifically, actions associated with a direction (e.g. movement, block placement, jumping, etc.), had a small probability ($0.3$) of moving in another random direction.


\subsection{Results}

\begin{table}
\centering
\caption{Expert Affordance Results: Avg. Number of Bellman Updates per converged policy}
\begin{tabular}{ l || c c | c c | c  c }
  & VI & A-VI & RTDP & A-RTDP & SG & A-SG \\ \hline
  \texttt{4BRIDGE} 		&	71604 		& 	{\bf 100} 		& 	836 		& 	{\bf 152} 		&	1373 	& 	{\bf 141}  			\\
  \texttt{6BRIDGE} 		&	413559 	 	& 	{\bf 366}  		& 	4561 	& 	{\bf 392} 		&	28185 	& 	{\bf 547}  \\
  \texttt{8BRIDGE} 		&	1439883 		& 	{\bf 904}		& 	18833	& 	{\bf 788}		&	15583	& 	{\bf 1001} 			\\
  \texttt{DOORB} 		&	861084 	 	& 	{\bf 4368}		& 	12207	& 	{\bf 1945}		&	6368		& 	{\bf 1381}	\\
  \texttt{LAVAB}  		&	413559 		& 	{\bf 366}	 	& 	4425 	& 	{\bf 993}  		&	25792	& 	{\bf 597}   \\
  \texttt{TUNNEL}  		&	203796 		& 	{\bf 105}		& 	26624	& 	{\bf 145}  		&	5404 	& 	{\bf 182}	\\
  \texttt{GOLD}  		&	16406		& 	{\bf 962}		& 	7738 	& 	{\bf 809}  		&	7412 	& 	{\bf 578}
\end{tabular}
\label{table:blocks}	
%  \end{tabular}
\label{table:hard-results}
\end{table}

Table~\ref{table:hard-results} shows the number of bellman updates required when solving the OO-MDP with RTDP (left column)
compared to solving the OO-MDP with an Affordance-Aware RTDP (right column).  The
affordance aware planner significantly outperformed its unaugmented
counterpart in all of these experiments. This
result demonstrates that affordances prune away many useless action in
these block building, block destruction, and gold smelting types of
tasks. 


% ====== Section: Learning-Affordances ======
\section{Learning-Affordances}
\label{sec:learning-affordances}

We have demonstrated that providing an (OO-)MDP solver with
a knowledge base of affordances can lead to dramatic speed ups in
planning. However, relying on experts to hand craft affordances removes autonomy
from the agent and places dependence on an expert. Instead we would like to have the agent learn these affordances
through experience to remove this strict dependence. We propose
a methodology for learning affordances directly, with some preliminary results
indicating the effectiveness of the system.

\subsection{Learning Process}

First, we modify our original formalism to account for the lack of expert knowledge. If we used the same formalism as in the expert
case, our learned affordances would often completely eliminate actions. Since the
learned affordances are more prone to make mistakes we cannot prune actions in this extreme way,
as we will lose optimality guarantees of the OO-MDP solver.

\begin{algorithm}
  \caption{$\Delta_i.getActions(s)$}
  \begin{algorithmic}[1]
    \State $\lambda \leftarrow DirMult(\Delta_i.\alpha)$
    \State $N \leftarrow Dir(\Delta_i.\beta)$
    \For {$1$ to $N$}
    \State $\Delta_i.\mathcal{A}' \leftarrow \lambda$
    \EndFor \\
    \Return $\Delta_i.\mathcal{A}'$
  \end{algorithmic}
  \label{alg:prune_actions}
\end{algorithm}

Instead, for a given state, we solve for the probability of getting a particular action set $\mathcal{A}^*$, and approximate sampling
from this distribution. This ensures that in the limit, it is possible to apply each action in each state.
\begin{equation}
\text{Pr}(\mathcal{A}^* | s, \Delta_0 \dots \Delta_N)
\end{equation}
We let know that each affordance contributes a set $\mathcal{A}'$ in each state:
\begin{align}
\text{Pr}(\mathcal{A}'_0 \cup \mathcal{A}'_N | s, \Delta_0 \dots \Delta_N)
\intertext{We approximate this term assuming the sets $\mathcal{A}_i'$
  are disjoint:} \sum_i \text{Pr}(\mathcal{A}'_i | s, \Delta_i)
\end{align}

For each affordance, to get an action set $\mathcal{A}'$, we form a Dirichlet Multinomial
over actions ($\lambda$), and a Dirichlet over the size ($N$) of each action set:

\begin{equation}
\text{Pr}(\lambda \mid \alpha) = DirMult(\alpha) \\
\end{equation}
\begin{equation}
\text{Pr}(N \mid \beta) = Dir(\beta)
\end{equation}

For each affordance we sample from our distribution over action set size to get a candidate action set size. We then
take that many samples from our distribution over actions to get a candidate action set $\mathcal{A}'$.

\begin{equation}
\text{Pr}(\mathcal{A}_i \mid s,\Delta_i) = \text{Pr}(\mathcal{A}_i' \mid N, \lambda) = \text{Pr}(\lambda \mid \alpha) \cdot \text{Pr}(N \mid \beta)
\end{equation}
\subsection{Computing $\alpha$ and $\beta$}

We require that an expert provide a set $\mathcal{P}$ of propositional functions
for the domain of relevance (i.e. Minecraft). Additionally, they must specify a set
$\mathcal{G} \subset \mathcal{P}$, that indicates which PropositionalFunctions may serve as goals.
We form a set of candidate affordances $\Delta$ with every combination of $\langle p, g \rangle$, for $p \in \mathcal{P}$ and $g \in \mathcal{G}$.

Then, we randomly generate a large number of small state spaces (typically on the order of several thousand), annotated with their
lifted goal description $g \in \mathcal{G}$. We solve the OO-MDP in each state space and get an optimal policy $\pi_j$. For each optimal policy,
we count the number of policies that used each action when each affordance was activated\footnote{An affordance is `activated' when its predicate is true
and the lifted goal description $g$ matches the agent's current goal}. These counts represent $\alpha$. Then, we count the number of unique actions
used by each policy, representing $\beta$.

\subsection{Experiments}

We tested our learning procedure on several simple worlds of varying size. We compared the performance of RTDP solving the OO-MDP
in each of these worlds with (1) No affordances, (2) Learned affordances, and (3) Expert provided affordances. We generated 100 simple state
spaces to learn on, each a $3\times3\times3$ world with randomized features based on the features of the agent's actual state space. As with
the expert affordance experiments, RTDP terminated when the maximum change in the
value function was less than 0.01 for five consecutive policy
rollouts. We set the reward function to $-1$ for all transitions and the discount
factor to $\lambda = 0.99$.

\subsection{Results}

Table \ref{table:learned-results} indicates the average number of bellman updates required by RTDP to solve the OO-MDP
in each of the four candidate worlds. The learned affordances clearly improved on standard RTDP by a significant margin, though
there is still a substantial gap between the learned affordance performance and that of the expert affordances. This indicates that
there is a lot of room for improvement in the learning process.

\begin{table}
\centering
\caption{Learned Affordance Results: Avg. Number of Bellman Updates per converged policy}
\begin{tabular}{ l || c c c }
  & No Affordances & Learned & Expert  \\ \hline
  \texttt{Tiny World}  		& 	879		&	576	&	 94	\\
  \texttt{Small World}  	& 	1460		&	1159	&	321  \\
  \texttt{Medium World}  	& 	3993		&	2412	&	693  \\
  \texttt{Large World}  	& 	8344		&	5100	&	1458
\end{tabular}
\label{table:learned-results}
\end{table}

% ====== Section: Related Work ======
\section{Related Work}
\label{sec:related-work}

In this section, we discuss the differences between
affordance-aware planning and other forms of knowledge that
have been used to accelerate planning.  Specifically, we discuss
temporally extended actions, logical methods, heuristics, and related action pruning
work.

% --- Subsection: Temporarily Extended Actions ---
\subsection{Temporarily Extended Actions}
Temporally extended actions are actions that the agent can
select like any other action of the domain, except executing them
results in multiple primitive actions being executed in
succession. Two common forms of temporally extended actions are {\em
  macro-actions}~\citep{hauskrecht98}
%   \jmnote{looking at the citaiton you added for macros Stefanie, I think those are still ``options''
%  rather than fixed primitive action sequences. Unfortunatley the term is overloaded a lot.}
  ~and {\em
  options}~\citep{sutton99}. Macro-actions are actions that always
execute the same sequence of primitive actions. Options are defined
with high-level policies that accomplish specific sub tasks. For
instance, when an agent is near a door, the agent can engage the
`door-opening-option-policy', which switches from the standard
high-level planner to running a policy that is hand crafted to open
doors. An option $o$ is defined as follows:

$o\ =\ \langle \pi_0, I_0, \beta_0\rangle$, where:
\begin{align*}
&\pi_0 : {\cal S} \times {\cal A} \rightarrow [0,1] \\
&I_0 : {\cal S} \rightarrow \{0,1\} \\
&\beta_0 : {\cal S} \rightarrow [0,1]
\end{align*}

Here, $\pi_0$ represents the {\it option policy}, $I_0$ represents
a precondition, under which the option policy may initiate, and 
$\beta_0$ represent the post condition, which determines which 
states terminate the execution of the option policy.

Although the classic options framework is not generalizable to different state spaces,
creating {\em portable} options is a topic of active research~\citep{konidaris07,konidaris2009efficient,Ravindran03analgebraic,croonenborghs2008learning,andre2002state,konidaris2012transfer}.

Although temporally extended actions are typically used
because they represent action sequences (or sub policies) that are often useful to solving
the current task, they can sometimes have the paradoxical effect
of increasing the planning time because they increase the number of actions that must be explored.
For example, deterministic planning algorithms that successfully make use of macro-actions often avoid the potential increase
in planning time by developing algorithms that restrict the set of macro-actions to a small set that is expected to improve planning time for the problem~\citep{Botea:2005kx,Newton:2005vn} or by limiting the use of macro-actions to certain conditions
in the planning algorithms like when the planner reaches heuristic plateaus (areas of the state space in which all successor states have the same heuristic value)~\citep{Coles:2007ys}. Similarly, it has been shown that the inclusion
of even a small subset of unhelpful options can negatively impact planning/learning time~\citep{Jong:2008zr}.

Given the potential for unhelpful temporally extended actions to negatively impact planning time, we believe combing affordances with temporally extended actions
may be especially valuable because it will restrict the set of temporally extended actions to those
useful for a task. In the future, we plan to explore the benefit from combining
these approaches.

% --- Subsection: Action Pruning ---
\subsection{Action Pruning}

Work that prunes the action space is the most similar to our affordance-aware planning.
Sherstov and Stone~\citep{sherstov2005improving} considered MDPs with a very large action set and for which the action
set of the optimal policy of a source task could be transferred to a new, but similar, target
task to reduce the learning time required to find the optimal policy in the target task. Since the actions
of the optimal policy of a source task may not include all the actions of the optimal policy
in the target task, source task action bias was reduced by randomly perturbing the value function
of the source task to produce new synthetic tasks. The action set transferred to the target task
was then taken as the union of the actions in the optimal policies for the source task and all the
synthetic tasks generated from it.

A critical difference between our affordance-based action set pruning and this action transfer
work is that affordances prune away actions on a state by state basis, where
as the learned action pruning is on per task level.
Further,
with lifted goal descriptions, affordances may be attached to subgoal planning for a significant
benefit in planning tasks where complete subgoal knowledge is known (or may be inferred).

Rosman and Ramamoorthy~\citep{rosman2012good} provide a method for learning action priors over a set of related tasks. Specifically, a Dirichlet distribution over actions was computed by extracting the frequency that each action was optimal in each state for each previously solved task. On a novel task learned with Q-learning, a variant of an $\epsilon$-greedy policy was followed in which the agent selected a random action according to the Dirichlet distribution an $\epsilon$ fraction of the time, and the action with the max Q-value the rest of the time. To avoid dependence on a specific state space, the a Dirichlet distribution was created for each observation-action pair (where the observations were task independent) instead of each state-action pair.

There are a few limitations of the actions priors work that affordance-aware planning does not possess: (1) the action priors can only be used with planning/learning algorithms that work well with an $\epsilon$-greedy rollout policy; (2) the priors are only utilized for fraction $\epsilon$ of the time steps, which is typically quite small; and (3) as variance in tasks explored increases, the priors will become more uniform. In contrast, affordance-aware planning can be used in a wide range of planning algorithms, benefits from the pruned action set in every time step, and the affordance defined lifted goal-description enables higher-level reasoning such as subgoal planning. However, in the future, the action set each affordance defines could be learned using a similar approach.

% --- Subsection: Temporal Logic ---
\subsection{Temporal Logic}

% --- Subsection: Heuristics ---
\subsection{Heuristics}
Heuristics in MDPs are used to convey information about the value of a given state or state-action pair with respect to the task being solved and typically take the form of either {\em value function initialization},
or {\em reward shaping}. For planning algorithms that estimate state-value functions, heuristics are often
provided by initializing the value function to values that are good approximations of the true value function. For example, initializing the value function to an admissible close approximation of the optimal value function has been shown to be effective for LAO* and RTDP, because it more greatly biases the states explored by the rollout policy to those important to the optimal policy~\citep{Hansen:1999qf}. Planning algorithms that estimate Q-values instead of the state value function may similarly initialize the Q-values to an approximation of the optimal Q-values. For instance, PROST~\citep{keller2012prost} creates a {\em determinized} version of a stochastic domain (that is, treating each action as if its most likely outcome always occurred), plans a solution in the determinized domain, and then initializes Q-values to the value of each action in the determinized domain.

Reward shaping is an alternative approach to providing heuristics in which the planning algorithm uses a modified version of the reward function that returns larger rewards for state-action pairs that are expected to be useful. Reward shaping differs from value function initialization in that it is not guaranteed to preserve convergence to an optimal policy unless certain properties of the shaped reward are satisfied~\citep{potshap} that also have the effect of making reward shaping equivalent to value function initialization for a large class of planning/learning algorithms~\citep{Wiewiora:2003fk}.

A critical difference between heuristics and affordances is that heuristics are highly dependent on the reward function and state space of the task being solved; therefore, different tasks require different heuristics to be provided, whereas affordances are state independent and transferable between different reward functions. However, if a heuristic can be provided, the combination of heuristics and affordances may even more greatly accelerate planning algorithms than either approach alone.


% ====== Section: Conclusion ======
\section{Conclusion}
\label{sec:conclusion}

We proposed a novel approach to representing knowledge in terms of
{\em affordances}~\citep{gibson77} that allows an agent to efficiently
prune its action space based on domain knowledge. This led to the 
proposal of affordance-aware planners, which improve on classic planners
by providing a significant reduction in the number of state-action pairs the
agent needs to evaluate in order to act optimally. We demonstrated the efficacy 
as well as the portability of the affordance model by comparing standard paradigm
planners to their affordance-aware equivalents in a series of challenging planning tasks in the Minecraft
domain.

Further, we designed a full learning process that allows an agent to autonomously learn useful affordances.
We provided preliminary results indicating the effectiveness of the learned affordances, suggesting that
the agent may be able to learn to tackle new types of problems on its own. Additionally, we proposed
a sampling method for use in rollout paradigm MDP solvers (such as RTDP), which to our knowledge has not been done before.

In the future, we hope to increase our coverage of Minecraft to tackle even more difficult planning problems, as well as
extending this approach beyond the Minecraft domain and onto actual robots, and other games (i.e. Atari). We also hope
to incorporate approaches to learning subgoals from text for use in conjunction with learning affordances.

\section{RSS citations}

%Please make sure to include \verb!natbib.sty! and to use the
%\verb!plainnat.bst! bibliography style. \verb!natbib! provides additional
%citation commands, most usefully \verb!\citet!. 
%  
%\subsection{RSS Hyperlinks}
%
%This year, we would like to use the ability of PDF viewers to interpret
%hyperlinks, specifically to allow each reference in the bibliography to be a
%link to an online version of the reference. 
%As an example, if you were to cite ``Passive Dynamic Walking''
%\cite{McGeer01041990}, the entry in the bibtex would read:
%
%{\small
%\begin{verbatim}
%@article{McGeer01041990,
%  author = {McGeer, Tad}, 
%  title = {\href{http://ijr.sagepub.com/content/9/2/62.abstract}{Passive Dynamic Walking}}, 
%  volume = {9}, 
%  number = {2}, 
%  pages = {62-82}, 
%  year = {1990}, 
%  doi = {10.1177/027836499000900206}, 
%  URL = {http://ijr.sagepub.com/content/9/2/62.abstract}, 
%  eprint = {http://ijr.sagepub.com/content/9/2/62.full.pdf+html}, 
%  journal = {The International Journal of Robotics Research}
%}
%\end{verbatim}
%}
%\noindent
%and the entry in the compiled PDF would look like:
%
%\def\tmplabel#1{[#1]}
%
%\begin{enumerate}
%\item[\tmplabel{1}] Tad McGeer. \href{http://ijr.sagepub.com/content/9/2/62.abstract}{Passive Dynamic
%Walking}. {\em The International Journal of Robotics Research}, 9(2):62--82,
%1990.
%\end{enumerate}
%%
%where the title of the article is a link that takes you to the article on IJRR's website. 
%
%
%Linking cited articles will not always be possible, especially for
%older articles. There are also often several versions of papers
%online: authors are free to decide what to use as the link destination
%yet we strongly encourage to link to archival or publisher sites
%(such as IEEE Xplore or Sage Journals).  We encourage all authors to use this feature to
%the extent possible.



\section*{Acknowledgments}

%% Use plainnat to work nicely with natbib. 
\bibliographystyle{plainnat}
\bibliography{main}

\end{document}


