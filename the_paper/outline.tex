\documentclass[a4paper]{article}

\usepackage[english]{babel}
\usepackage[utf8x]{inputenc}
\usepackage{amsmath}
\usepackage{amsfonts}
\usepackage{amssymb}
\usepackage{graphicx}
\usepackage{aaai}
\usepackage{listings}
\usepackage{float}
\usepackage{bbm}
\usepackage{wasysym}
\usepackage[colorinlistoftodos]{todonotes}

\newenvironment{courier}{\fontfamily{pcr}\selectfont}

\title{Planning with Affordances}
\author{Dave Abel, Gabriel Barth-Maron, and Stefanie Tellex \\ \\ Department of Computer Science, Brown University \\ \\ \begin{courier}\{dabel,gabrielbm,stefie\}@cs.brown.edu\end{courier}}
\date{}
\begin{document}
\maketitle

\begin{abstract}
lorum ipsum...
\end{abstract}

\section{I. Introduction}
\begin{enumerate}
\item Planning

\item Reinforcement Learning

\item Value Iteration

\item Minecraft

\item Affordance Formalism
\end{enumerate}

\section{II. Related Work}
\begin{enumerate}
\item Partial Order Planning

\item Branavan (learning sub goals through text)

\item Grounds and Kudenko, RL + symbolic planning

\item Koppola and Saxena (using affordances for \_\_\_)

\item Steedman, Formalizing Affordances

\item OOMDP

\item Teaching robot grounded relational symbols (toussaint)

\item RRTs
\end{enumerate}
\section{III. Background}
\begin{enumerate}
\item Reinforcement Learning
\item Planning in general (PDDL, STRIPS)
\item Partial Order Planning
\item Affordances and Gibson
\item Minecraft, Subgoals
\end{enumerate}
\section{IV. Model}

\begin{enumerate}
\item Affordance Formalism
\end{enumerate}

\section{V. Learning}
?

\section{VI. Evaluation}

\begin{enumerate}
\item Complexity
\item Proof(s) of optimality?
\item Empirical data on scenarios in Minecraft
\item Other baselines? (RRT, A*, Random, etc)
\end{enumerate}

\section{VII. Conclusion}
lorum ipsum...

\section{Acknowledgments}
We would like to thank these peeps:
\section{References}
lorum ipsum... (see project wiki)

\end{document}