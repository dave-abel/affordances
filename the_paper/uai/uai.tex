\documentclass[]{article} 
\usepackage{proceed2e}

\usepackage[numbers,sort]{natbib}
\usepackage{amsmath}

\title{Planning with Affordances}
\newcommand{\ignore}[1]{}

\begin{document}
\author{}
\maketitle

\begin{abstract}
Current methods for exactly solving decision-making under uncertainty
require exhaustive enumeration of all possible states and actions,
leading to exponential run times, leading to the well-known ``curse of
dimensionality.''  Approaches to address this problem by providing the
system with formally encoded knowledge such as options or
macro-actions, still fail to prevent the system from considering many
actions which seem obviously irrelevant for a human partner.  To
address this issue, we introduce a novel approach to representing
knowledge about how to plan in terms of {\em
  affordances}~\citep{gibson77}.  Our affordance formalism and
associated planning framework allows an agent to efficiently prune its
action space based on domain knowledge.  This pruning significantly
reduces the number of state/action pairs the agent needs to evaluate
in order to act optimally.  We demonstrate our approach in the
Minecraft domain on several planning and building tasks, showing a
significant increase in speed and reduction in state-space exploration
compared to partial order planning, options, and macro-actions.
\end{abstract}

\section{INTRODUCTION}

As robots move out of the lab and into the real world, planning
algorithms need to be able to scale to domains of increased noise,
size, and complexity.  A classic formalization of this issue is the
sequential decision making problem, where increases in problem size
and complexity directly correspond to an explosion in the state-action
space. Current approaches to solving sequential decision making
problems cannot tackle these problems as the state-action space
becomes large~\citep{Grounds2005}.

There is a strong need for a generalizable form of knowledge that,
when coupled with a planner, is capable of solving problems in these
massive domains. Humans provide an excellent existence proof for such
planning, as we are capable of searching over an immense number of
possible actions when presented with a goal.  One approach to
explaining how humans solve this planning problem is by focusing on
problem-specific aspects of the environment which focus the search
toward the most relevant and useful parts of the state/action space.
Formally, \citet{gibson77} proposed to define this intuition as an
{\em affordance}, ``what [the environment] offers [an] animal, what
[the environment] provides or furnishes, either for good or ill.''
Additionally, roboticists have recently become interested in
leveraging affordances for perception and prediction of
humans~\citep{koppula13a, koppula13b}. In this paper we will formalize
the notion of an affordance as a piece of planning knowledge provided
to an agent operatingin a Markov Decision Process
(MDP)~\citep{kaelbling99}.  We demonstrate that, like an option or
macro-action, an affordance provides additional information to the
agent, enabling more efficient planning.  However, unlike previous
approaches, an affordance enables more significant speedups by
reducing the size and branching-factor of the search space, enabling
an agent to focus its search on the most relevant part of the problem
at hand.  This approach means that a {\em single} set of affordances
provides general domain knowledge, becoming relevant just when the
agent reasons that it needs to pursue a particular subgoal.  


\begin{figure}
\parbox{1\linewidth}{
XXXXX~\\
XXXXX~\\
XXXXX~\\
}
\caption{Figure demonstrating the intuition of the approach.  Maybe a
  minecraft scene annotated with the affordances used to build a
  bridge?\label{fig:example}}
\end{figure}

\section{BACKGROUND}

\subsection{SUBGOALS}
% % Intuition
Subgoal planning leverages the intuition that certain goals in planning domains may only be brought about if certain preconditions are first satisfied. For instance, in the Minecraft domain, one must be in possession of grain in order to bake bread. In Branavan et. al, they explore learning subgoals and applying them in order to plan through a variety of problems in Minecraft.

% Formalism
Formally, in subgoal planning, the agent is given a pair of predicates:

\[ 
<x_k, x_l >
\]

where $x_l$ is the effect of some action sequence performed on a state in which $x_k$ is true. Thus, subgoal planning requires that we perform high-level planning in subgoal space, and low-level planning to get from subgoal to subgoal.

\[
\boxed{\text{Running Example}}
\]

% Maybe consider inserting Branavan's chart (or similar), or discuss how the search in subgoal space works?

{\bf Problem 1: Loss of generality}  One important thing to note about subgoals that {\it are} general enough to enhance an agent's planning abilities in a wide variety of state spaces is that they propose {\it necessary} claims about the domain that the agent occupies. If the subgoals are {\it contingent} (i.e. true in some state spaces of the domain but not in others), then they can be shown to completely lose their scalability. For instance, consider the task in \texttt{BRIDGEWORLD}, in which the agent must place a block in the trench that separates the agent from the goal. The subgoal $<blockInTrench(), agentAtGoal()>$ might be a perfectly useful goal in {\it this} planning scenario, but an adversary could easily come up with thousands of worlds in which such a subgoal would completely derail the agent's planner. In order for subgoals to be useful, they must be necessary claims about the domain, otherwise, one can always come up with a counter world (by definition of necessary). 

{\bf Problem 2: Granular Planning} However, this poses a serious problem for subgoal planning, as the vast majority of tasks in mine craft are not so easily broken into useful, necessary subgoals. Movement for instance is particularly difficult, as the subgoals for movement must be necessary claims (i.e. $<agentOneAwayFromGoal(), agentAtGoal()>$), otherwise, they can be broken. Unfortunately, coming up with such subgoals is not an easy task, and often the best we can do is to plan at such a low level that we lose any benefit of planning over subgoals to begin with

% Figure of granular planning w/ subgoals

% PseudoCode

\subsection{OPTIONS}

The options framework proposes incorporating high-level policies to accomplish specific sub tasks. For instance, when an agent is near a door, the agent can engage the `door-opening-option-policy', which switches from the standard high-level planner to running a policy that is hand crafted to open doors. An option $o$ is defined as follows:

$o\ =\ <\pi_0, I_0, \beta_0>$, where:

\begin{itemize}
\item[] $\pi_0 : (s,a) \rightarrow [0,1]$
\item[] $I_0 : s \rightarrow \{0,1\}$
\item[] $\beta_0 : s \rightarrow [0,1]$
\end{itemize}

Here, $\pi_0$ represents the {\it option policy}, $I_0$ represents a precondition, under which the option policy may initiate, and $\beta_0$ represent the post condition, which determines which states terminate the execution of the option policy.

As Konidaris and Barto point out, the classic options framework is not generalizable, as it does not enable an agent to transfer knowledge from one state space to another. Recently, Konidaris and Barto's \ignore{cite} expand on the classic options framework and allow for a more portable implementation of options. Still, though, planning with options requires either that we plan in a mixed space of actions {\it and} options (which blows up the size of the search space), or requires that we plan entirely in the space of options. Additionally, providing an agent with an option policy is a difficult task for a human designer (especially if we want an optimal policy, which we do).

\subsection{MACROACTIONS}

\[
\boxed{\text{Running Example}}
\]

\section{AFFORDANCES}

%% Intuition
% An affordance is what an environment, or the objects in it, offer the agent. For example, if the agent were trying to drink coffee then a mug would afford carrying liquid. However if instead the agent wanted to secure some papers outside on a windy day, then the mug could afford weighing the papers down (i.e. affords using it as a paper weight). Often, objects may be defined in terms of what they afford, such as a bridge - suppose there is a log lying horizontally across a river; such a log provides passage over a river and could reasonably be referred to as a "bridge" simply because it affords crossing the river.

%% Formalism
Formally, an Affordance is defined as \vspace{1 mm} \\
{\it Aff} $ =\ <p,g>\ \longrightarrow \alpha$, where:

\begin{itemize}
\item[] $\alpha \subseteq \mathcal{A}$
\item[] $p : s \longrightarrow \{$0$, 1\}$
\item[] $g : s \longrightarrow \{$0$,1\}$
\end{itemize}

Where $\alpha$ is a subset of the agent's given set of actions $\mathcal{A}$, $p$ is a {\it precondition} that is a predicate over states, and $g$ is a {\it goal} or {\it subgoal} that is also a predicate over states.

The constituents that make up an Affordance parallel those of the other planning approaches discussed in the background section.

\[
\boxed{\text{Running Example}}
\]

The Affordance formalism introduced above and expanded on in this paper resolves the weaknesses of these other frameworks by limiting the complexity of the seed knowledge required of the designer, providing enough knowledge to limit the search space, and still maintains scalability.

We should be able to prove that given a ``good" set of subgoals the agent will be able to reach one after the other with high probability. Therefore, in the case that the agent cannot reach a subgoal there is likely a better one. The agent should then prompt for a more specific subgoal that will better allow it to reach the next one.

\section{EXPERIMENTS}

Task List, apply each planning system (Aff, O, SG) to all tasks.

% List from wiki: (we should pick our task set soon)

%Build a Tower
%Indestructible Wall w/ an opening
%U Shaped wall
%Find destructible part of wall
%Build stairs over wall
%Use Ladder to reach goal (already in world)
%Trench of varying width (only have 1 block) ?+? build tower on other side
%Path Planning (simple)
%Bridge
%Tunnell through wall
%Use door to get through wall
%Build something more complicated (a cube? a pyramid? a wall? -> something that restricts access to a goal i.e. makes it impossible to reach)
%Branavan scenarios
%Options four rooms
%Macroactions
% George lightroom example



\section{RESULTS}

% We would like a table like this:
\begin{tabular}{ l || c | c | c }
  & Affordances & Options & Subgoals \\
  \hline
  {\bf Task One} & 1 & 1 & 0 \\
  {\bf Task Two} & 1 & 1 & 1 \\
  {\bf Task Three} & 1 & 1 & 0 \\
\end{tabular}

% As well as some charts indicating search space size, #cycles, etc..

\section{CONCLUSION}

\bibliographystyle{plainnat}
\bibliography{main}  


\end{document}
